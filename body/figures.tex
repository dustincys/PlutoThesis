% !Mode:: "TeX:UTF-8" 

\BiChapter{图片的插入方法}{Methods of inserting figures}

\BiSection{研究生院的插图规范}{Figures inserting standard from graduate school}
图应有自明性。插图应与文字紧密配合,文图相符,内容正确。选图要力求精练,插图、照片应完整清晰。图中文字和数字等字号用宋体~5~号字。

机械工程图:采用第一角投影法,严格按照~GB4457---GB131-83《机械制图》标准规定。

数据流程图、程序流程图、系统流程图等按~GB1526-89~标准规定。

电气图:图形符号、文字符号等应符合附录~3~所列有关标准的规定。

流程图:必须采用结构化程序并正确运用流程框图。

对无规定符号的图形应采用该行业的常用画法。

坐标图的坐标线均用细实线,粗细不得超过图中曲线,有数字标注的坐标图,必须注明坐标单位。

照片图要求主题和主要显示部分的轮廓鲜明,便于制版。如用放大或缩小的复制品,必须清晰,反差适中。照片上应有表示目的物尺寸的标度。

引用文献图表必须标注出处。


\BiSubsection{图题及图中说明}{Captions and descriptions of figures}
每个图均应有图题(由图序和图名组成),图名在图序之后空一格排写。图序按章编排,如第~1~章第一个插图的图号为“图~1-1”等。
图题置于图下,硕士论文可只用中文书写,博士论文用中、英文两种文字居中书写,中文在上,要求中文用宋体~5~号字,英文用~Times New Roman 5~号字。有图注或其它说明时应置于图题之上。引用图应注明出处,在图题右上角加引用文献号。
图中若有分图时,分图题置于分图之下或图题之下,分图号用~a)、b)等表示。

图中各部分说明应采用中文(引用的外文图除外)或数字项号,各项文字说明置于图题之上(有分图题者,置于分图题之上)。

\BiSubsection{插图编排}{Layouts of illustrations}
插图之前,文中必须有关于本插图的提示,如“见图~1-1”、“如图~1-1~所示”等。插图与其图题为一个整体,不得拆开排写于两页。
插图处的该页空白不够排写该图整体时,则可将其后文字部分提前排写,将图移到次页。

\BiSection{\LaTeX~中推荐使用的图片格式}{Recommended figure format applied in \LaTeX}
在~\LaTeX~中应用最多的图片格式是~EPS(Encapsulated PostScript)格式,它是一种专用的打印机描述语言,常用于印刷或打印输出。
EPS~格式图片可通过多种方式生成,这里介绍一款功能强大的免费图片处理软件———\href{http://www.imagemagick.org/}{ImageMagick},
360~软件管家也提供此软件的下载。此软件可将其它格式图片转换为~EPS~格式图片,同时还可以锐化图片,使图片的局部清晰一些。

此软件对图片的格式转换操作都是在命令提示符(cmd.exe)中实现的,可以通过“开始$\to$运行$\to$输入~cmd$\to$回车”或
“开始$\to$程序$\to$附件$\to$命令提示符”找到它。在命令提示符下,首先采用“盘符命令”或“cd~命令”将当前目录改为待处理图片所在的目录,
在此目录下就可通过~convert~命令将图片转换为~EPS~格式,其命令的语法格式为

\noindent\verb|convert [可选参数] 原文件名.原扩展名 新文件名.eps|

\noindent 若~convert~命令中无可选参数,则将原来的图片格式直接转换为~EPS~格式,对图片不进行任何处理,这也是最常用的方法。
也可以选用可选参数,可选参数有很多选择,但最常用的有如下两个:

\verb|-sharpen radius{xsigma}|———此参数用来锐化图片,一般用在图片像素不高,需要提高图片清晰度的情况下。其中~radius~只能为整数,
它用来确定转换命令采取哪一种锐化算法,我们可以只取~radius~为~0;sigma~为所采取算法的锐化度,它的取值为~0.1--3~之间的任意一个浮点数,
数值越大,锐化程度也越大,通常取为~0.5--1~之间;x在参数中为分隔符。

\verb|-resize geometry|———此参数用来改变图片的大小,若图片的存储空间过大,可通过此命令缩小图片尺寸,但同时也将导致图片像素降低,
其具体用法请参见\href{http://www.imagemagick.org/script/command-line-options.php#resize}{-resize geometry的官方说明}。

除此之外,一些文字处理软件和科学计算软件也支持生成~EPS~格式的文件,请使用“另存为”功能查看某款软件是否能够将图片以~EPS~格式的形式保存。

\BiSection{单张图片的插入方法}{The method of inserting one single figure}
单张图片独自占一行的插入形式如图~\ref{golfer1}~所示。
\begin{figure}[htbp]
\centering
\includegraphics[width = 0.4\textwidth]{golfer}
\bicaption[golfer1]{}{打高尔夫球的人}{Fig.$\!$}{The person playing golf}\vspace{-1em}
\end{figure}

其插入图片的代码及其说明如下。
\vspace{1em}\noindent\hrule
\begin{lstlisting}
\begin{figure}[htbp]
\centering
\includegraphics[width=0.4\textwidth]{文件名(.eps)}
\bicaption[标签名(英文)]{}{中文标题}{Fig.$\!$}
          {English caption (首字母大写)}\vspace{-1em}
\end{figure}
\end{lstlisting}
\noindent\hrule
\begin{lstlisting}
figure环境的可选参数[htbp]表示浮动图形所放置的位置,h (here)表示当前位置,t (top)表示页芯顶部,b (bottom)表示页芯底部,p (page)表示单独一页。在word等软件中,图片通常插入到当前位置,如果当前页的剩余空间不够,图片将被移动到下一页,当前页就会出现很大的空白,其人工调整工作非常不便。由LaTeX提供的浮动图片功能,总是会按h->t->b->p的次序处理选项中的字母,自动调整图片的位置,大大减轻了工作量。
\centering命令将后续内容转换成每行皆居中的格式。
“\includegraphics”的可选参数用来设置图片插入文中的水平宽度,一般表示为正文宽度(\textwidth)的倍数。
\bicaption命令的使用需要调用ccaption宏包,它可以为图片或表格插入双语标题(博士学位论文要求),可选参数“标签名”为英文形式,一般不以图片或表格的数字顺序作为标签,而应包含一定的图片或表格信息,以便于文中引用(若图片、表格、公式、章节和参考文献等在文中出现的先后顺序发生了变化,其标注序号及其文中引用序号也会跟着发生变化,这一点是word等软件所不能做到的)。第4个参数中的“$\!$”表示-1/6个空铅宽度,这样可以缩小Fig.和Table与后面数字序号之间的水平距离。另外,图题或表题并不会因为分页而与图片或表格体分置于两页,章节等各级标题也不会置于某页的最底部,LaTeX系统会自动调整它们在正文中的位置,这也是word等软件所无法匹敌的。
注:硕士学位论文的图表只需要插入中文标题,因此需将\bicaption一句命令替换为如下两条命令(下同):
\caption{中文标题}
\label{标签名(英文)}
\vspace将产生一定高度的竖直空白,必选参数为负值表示将后续文字位置向上提升,参数值可自行调整。em为长度单位,相当于大写字母M的宽度。
引用方法:“见图~\ref{标签名(英文)}”、“如图~\ref{标签名(英文)}~所示”等。
\end{lstlisting}
\noindent\hrule\vspace{1em}
若需要将~2~张及以上的图片并排插入到一行中,则需要采用\verb|minipage|环境,如图~\ref{golfer2}~和图~\ref{golfer3}~所示。
\begin{figure}[htbp]
\centering
\begin{minipage}{0.4\textwidth}
\centering
\includegraphics[width=\textwidth]{golfer}
\bicaption[golfer2]{}{打高尔夫球的人}{Fig.$\!$}{The person playing golf}
\end{minipage}
\begin{minipage}{0.4\textwidth}
\centering
\includegraphics[width=\textwidth]{golfer}
\bicaption[golfer3]{}{打高尔夫球的人}{Fig.$\!$}{The person playing golf}
\end{minipage}\vspace{-1em}
\end{figure}

其代码如下所示。
\vspace{1em}\noindent\hrule
\begin{lstlisting}
\begin{figure}[htbp]
\centering
\begin{minipage}{0.4\textwidth}
\centering
\includegraphics[width=\textwidth]{文件名}
\bicaption[标签名]{}{中文标题}{Fig.$\!$}
          {English caption}
\end{minipage}
\begin{minipage}{0.4\textwidth}
\centering
\includegraphics[width=\textwidth]{文件名}
\bicaption[标签名]{}{中文标题}{Fig.$\!$}
          {English caption}
\end{minipage}\vspace{-1em}
\end{figure}
\end{lstlisting}
\noindent\hrule
\begin{lstlisting}
minipage环境的必选参数用来设置小页的宽度,若需要在一行中插入n个等宽图片,则每个小页的宽度应略小于(1/n)\textwidth。
\end{lstlisting}
\noindent\hrule

\BiSection{具有子图的图片插入方法}{The method of inserting figures with subfigures}

图中若含有子图时,需要调用~subfigure~宏包。博士学位论文规范要求不止总图的标题为中英文形式,其各个子图也应具有中英文形式的标题。
然而~ccaption~宏包却无法实现子图的中英文标题功能,这里采用对\verb|\subfigure|命令进行嵌套的方法来实现子图的中英文标题功能,如图~\ref{golfer4}~所示。

\begin{figure}[htbp]
\centering
\subfigure{\label{golfer41}}\addtocounter{subfigure}{-2}
\subfigure[The person playing golf]{\subfigure[打高尔夫球的人~1]{\includegraphics[width=0.4\textwidth]{golfer}}}
\subfigure{\label{golfer42}}\addtocounter{subfigure}{-2}
\subfigure[The person playing golf]{\subfigure[打高尔夫球的人~2]{\includegraphics[width=0.4\textwidth]{golfer}}}
\bicaption[golfer4]{}{打高尔夫球的人}{Fig.$\!$}{The person playing golf}\vspace{-1em}
\end{figure}

其代码及其说明如下。
\vspace{1em}\noindent\hrule
\begin{lstlisting}
\begin{figure}[htbp]
\centering
\subfigure{\label{第1个子图标签名}}\addtocounter{subfigure}{-2}
\subfigure[The 1st subfigure caption]{\subfigure[第1个子图标题]
          {\includegraphics[width=0.4\textwidth]{文件名}}}
\subfigure{\label{第2个子图标签名}}\addtocounter{subfigure}{-2}
\subfigure[The 2nd subfigure caption]{\subfigure[第2个子图标题]
          {\includegraphics[width=0.4\textwidth]{文件名}}}
\bicaption[总标签名]{}{中文总标题}{Fig.$\!$}{The total caption}
\vspace{-1em}
\end{figure}
\end{lstlisting}
\noindent\hrule
\begin{lstlisting}
\addtocounter把指定的值加到计数器上,这里是对subfigure计数器进行减2操作。这是因为每插入1个子图,就调用3次\subfigure命令,第1次调用\subfigure命令用来生成紧随其后所插入子图的标签,而之后的双层嵌套调用\subfigure命令用来插入子图并生成该子图的中英文标题。因此,每插入1张子图,subfigure计数器的值就自动加3,为了使得子图的序号能每次加1,则需在每插入1张子图前手动把subfigure计数器的值减2。
\subfigure命令的双层嵌套使用可用来生成中英文标题,其内层\subfigure命令用来插入子图并生成中文标题,外层\subfigure命令将插入的子图和中文标题作为一个整体,生成这个整体的英文标题,因此英文标题会置于中文标题的下面。
硕士学位论文只需要中文标题,其代码如下:
\begin{figure}[htbp]
\centering
\subfigure[第1个子图标题\label{第1个子图标签名}]
          {\includegraphics[width=0.4\textwidth]{文件名}}
\subfigure[第2个子图标题\label{第2个子图标签名}]
          {\includegraphics[width=0.4\textwidth]{文件名}}
\caption{中文总标题}\label{总标签名}
\vspace{-1em}
\end{figure}
引用方法:总图的引用方法同本章第1节,子图的引用方法用\ref{第n个子图标签名}来代替。
\end{lstlisting}
\noindent\hrule\vspace{1em}

子图的引用示例:如图~\ref{golfer41}~和图~\ref{golfer42}~所示。

若想获得插图方法的更多信息,请参见网络上的~\href{ftp://ftp.tex.ac.uk/tex-archive/info/epslatex.pdf}{Using Imported Graphics in \LaTeX and pdf\LaTeX}~文档。 
